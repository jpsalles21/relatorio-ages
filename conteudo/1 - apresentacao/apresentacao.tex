\chapter[APRESENTAÇÃO DA TRAJETÓRIA DO ALUNO]{APRESENTAÇÃO DA TRAJETÓRIA DO ALUNO}

Minha trajetória na área de Software inicia antes da faculdade. Aos
meus 11 anos jogava Minecraft e era fascinado em como os plugins
deixavam o jogo muito mais atrativo e diferente e comecei a buscar na
internet maneiras de editar e criar um plugin próprio. Foi neste momento que
me deparei com a linguagem Java, no começo foi estranho e diferente, mas
consegui concluir alguns projetos e fiquei muito feliz por isso.
\par Não continuei na área e minhas pretensões eram outras, mas na
hora de escolher meu curso, movido pela paixão e por aquele sentimento do passado, resolvi entrar no curso de
Engenharia de Software no segundo semestre de 2022. Desde então
procuro evoluir como frontend e sigo um entusiasta da área.
\par Em maio de 2023, participei da Hackatona de Engenharia de
Software na PUCRS, onde deveríamos resolver uma solução com o tema
“Como utilizar a Inteligência Artificial para promover o bem social?”. Nosso
grupo apresentou a solução da Comunicação Universal, projeto que
consiste em um tradutor de libras em tempo real. Obtivemos o segundo
lugar entre 14 equipes. \ac{ages}.

    