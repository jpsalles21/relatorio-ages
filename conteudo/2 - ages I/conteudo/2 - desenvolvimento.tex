\hypersetup{
    colorlinks=true,
    linkcolor=blue,
    filecolor=blue,      
    urlcolor=blue,
    citecolor=blue,
}


\section[Desenvolvimento do Projeto]{Desenvolvimento do Projeto}

\subsection{Repositório do Código Fonte do Projeto}
Os repositórios foram divididos entre backend, frontend, e wiki do projeto.

Repositório para o \href{https://tools.ages.pucrs.br/informativo-paraimigrantes/informativo-para-imigrantes-backend}{backend}

Repositório para o \href{https://tools.ages.pucrs.br/informativo-paraimigrantes/informativo-para-imigrantes-frontend}{frontend}

Repositório para \href{https://tools.ages.pucrs.br/informativo-paraimigrantes/wiki}{wiki}



\subsection{Banco de Dados Utilizado}
  No banco de dados, estamos utilizando o SQL Server. Este foi
implentando nas Sprints finais e, eu, como AGES I não obtive partcipação
simbólica para a implementação.

\subsection{Arquitetura Utilizada}
 O projeto usa Java 11 e Spring Boot e possui três camadas: Controller
(apresentação), Service (lógica de negócios) e Repository (acesso a dados). Os
módulos incluem configurações, controle de rotas, objetos de transporte de dados,
entidades de banco de dados, enums, mensagens, comunicação com o banco de
dados, segurança, regras de negócios e utilitários. No frontend, há estruturas para
recursos, componentes, telas, rotas, serviços, utilitários, hooks e acesso a APIs.
Essas estruturas organizam o projeto de forma eficiente.

\subsection{Protótipos das Telas Desenvolvidas}
Utilizamos o Figma para o protótipo de telas, onde mostramos para o cliente a
ideia inicial do visual do aplicativo e como implementaríamos as funções que
eles gostariam que o programa tivesse.
\begin{figure}[H]
    \centering
    \small
    \caption{Protótipo no Figma}
    \includegraphics[width=1\linewidth]{conteudo/2 - ages I/conteudo/figures/figma1.png}
    Fonte: figma do projeto
    \label{fig:projeto-time}
\end{figure}

\subsection{Tecnologias Utilizadas}
Na primeira aula, iniciamos a discussão de quais tecnologias
seriam utilizadas. Pelo backend, foi definido que utilizaríamos Java,
com o framework Spring Boot, visto que o grupo tudo tinha
familiaridade com a linguagem, já que esta é ensinada nas disciplinas
da faculdade. Já no lado do frontend, decidimos utilizar TypeScript
com o framework React Native, já que o aplicativo seria desenvolvido
tanto para iOS, tanto para Android.