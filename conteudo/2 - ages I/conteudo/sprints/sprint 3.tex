\subsection{Sprint 3}
Dando continuidade no projeto, na Sprint 3 permaneci no
frontend e fiquei encarregado de criar e estilizar as tabs de navegação, criar
a tela de meus programas e o dropdown para tipos de programas. Após
quase 2 meses de projeto, já estava confiante e tinha muito mais
conhecimento nas linguagens que estávamos utliziando, por isso não obtive
muitas dificuldades em executar as tarefas que foram passadas.
Nas tabs de navegação, utilizei o Material Top Tabs Navigator, do
React Navigation, e as estilizei de maneira que ficassem o mais fidedignas
com as desenvolvida no Figma. Criei a tela de meus programas, apenas
com os mocks para que depois, quando necessário, fizessem a integração.
Nesta Sprint também fiquei responsável por um débito técnico na tela de de
cadastro de programas, onde o usuário selecionava as informações do
programa, faltava um dropdown para escolher se o tipo de programa era
“Ensino Básico, Ensino Superior ou uma ONG”. Todas as tarefas foram
resolvidas e, até o momento considerava essa a sprint onde me senti mais
pronto e realizado.
Também entendi muito mais os conceitos do Git e comecei a subir
minha branch e commits mais cedo, assim dando margem para que os
AGES III pudessem analisar meu código e me passar o feedback caso
alguma coisa não esteja como o esperado. 