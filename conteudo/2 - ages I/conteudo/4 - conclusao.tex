\section[Conclusão]{Conclusão}

O projeto agradou muito aos stakeholders, nos parabenizando e
ficando encantados com todo o trabalho feito. Eu também gostei, foi muito
interessante entender como funcionava todo o desenvolvimento de um
projeto e claro, ter uma parte que foi feita por mim é muito gratificante e
empolgante para quem está apenas iniciando na área.
\par Tive um desenvolvimento muito bom como AGES I, cheguei com
conhecimento sobre frontend praticamente nulo e consegui ter uma boa
curva de aprendizado. Não sabia nada sobre integração de backend com
frontend, como utilizar o Postman e com um pouco de conhecimento sobre
versionamento e agora posso dizer que entendo bastante sobre estes
conceitos. Também me destaco pela parte não técnica, não pude participar
de nenhuma “daily” de terça de maneira síncrona no discord, pois estava
cursando a disciplina de Engenharia de Requisitos no mesmo horário, mas
sempre atualizei meus progressos e impedimentos pelo canal de texto no
dia. Foi muito agregador entender como funciona a parte mais “burocrática”
de desenvolver um software, as dailys, retros e plannings me
proporcionaram experiências que não teria vivido se não fosse a AGES. Por
tanto, agora saio muito mais preparado para embarcar no mercado de
trabalho e começar a aplicar meus conhecimentos adquiridos nesta etapa.
\par O projeto por si só foi muito bom, não tenho experiências passadas para comentar
com mais detalhes, mas a organização foi excelente. Acredito que o trabalho do
AGES IV foi muito bom, mas aqui destaco a eficiência de todos os AGES III que
contribuíram muito e foram essenciais no andamento do projeto e no crescimento
dos AGES I e II. Por contrapartida, acredito que o projeto pecou um pouco na hora
de designar as tarefas, pois não ficava muito explícito o que deveria ser feito,
resultando em dúvidas e mais trabalho para os AGES III. Também creio que o
projeto seria melhor se fosse desenvolvido web, entretanto esta é uma questão
definida pelo stakeholder
