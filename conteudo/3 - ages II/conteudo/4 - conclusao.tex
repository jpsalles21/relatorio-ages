\section[Conclusão]{Conclusão}

O projeto Let Me Trial foi um marco totalmente diferente para mim no quesito de desenvolver software, já que, por falta de alunos com maior conhecimento em \textit{frontend}, tive que fazer diversas coisas de maneira solitária, mas que contribuíram de maneira significativa para meu crescimento dentro da área de desenvolvimento de software. Por consequência, acredito que consegui constribuir de uma forma muito boa devido aos limites tanto dos meus conhecimentos, quanto dos colegas. Tive um bom relacionamento e uma boa comunicação com todos do projeto, assim como fui proativo e sempre estive proposto a ajudar todos com dúvidas e tentar resolver problemas que foram aparecendo, além de manter a pontualidade em todas as reuniões e sempre com a presença em dia.
\par Definitivamente este foi o projeto, até então, em que eu mais aprendi em relação a tudo, tanto quanto a parte do desenvolvimento de código, quanto a importância dos processos de software. Visto que o grupo todo conseguiu entender como tudo funcionava na metodologia ágil, além de uma boa contribuição de todos sempre que possível, o time foi harmônico e cada um teve  seu momento de crescer e disparar no projeto e no final foi bonito e inspirador ver o tanto que evoluíram. 
Aprendi muito mais sobre \textit{frontend} e suas técnicas, que por mais que o nível inicial de conhecimento de todos era mais baixo, foi o maior ponto positivo do projeto, já que todos estiveram focados em aprender e repassar o conhecimento que foi adquirido em pouco tempo. Consegui, sempre que possível, diagnosticar minha situação nais dailys e participei ativamente das discussões, principalmente voltado para a área da interface de usuário. O projeto poderia ter tido um início mais calmo e mais organizado, fomos desfavorecidos no começo pelas retros exageradas e muito calorosas, que nos fez perder uma planning que seria muito importante, já que seria a primeira dessas na parte de desenvolvimento de código. Faltou uma participação maior dos AGES III, visto que nosso projeto, por exemplo, ainda está sendo apresentado em \textit{localhost}, além de deixar uma sobrecarga muito grande nas costas dos nossos AGES IV.
