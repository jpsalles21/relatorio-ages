\subsection{Sprint 0}

Na sprint 0, depois da visita dos stakeholders, definimos metas e
como faríamos para manter o projeto. Ficou acordado que teríamos dailys
assíncronas no Discord todos os dias, onde deveríamos reportar o que
fizemos, o que faremos e se havia ou não algum impedimento. Depois disso,
foi definida qual linguagens e frameworks utilizaríamos no projeto, ficando
acordado pelo Java 21 com Spring Boot e React com TypeScript e Next.
\par Após toda a cerimônia, comecei a desenvolver uma tela de login, que
serviu como modelo para outras, até chegarmos no produto final. Sempre
que pude, estive dando ideias e produzindo ou modificando algum
componente, para que ficasse de acordo com o que os clientes esperavam.
Nosso primeiro encontro da sprint foi produtivo e conseguimos alinhar o que
faltava para entregarmos ao cliente todo o fluxo e a interface do usuário.
\par Depois de semamans produzindo, nosso cliente chegou e gostou
muito do que foi proposto no quesito fluxo e beleza das telas. Entretanto,
tínhamos uma dúvida pertinente, que era como seria feito o filtro de estudos
e, após uma aula toda, chegamos em um consenso.
\par Após isso, fizemos a retro, que, por falta de comando, acabou
tomando muito tempo com dicussões inúteis que não precisariam de tanta
minutagem para serem debatidas, o que nos custou a aula toda. Sendo
assim não foi possível realizar a planning na sala presencialmente. Uma
pena, já que eu considero a planning um dos momentos mais importantes
para uma sprint.
