\subsection{Sprint 1}

Começamos a sprint 1, sem planning e sem repositórios do frontend,
visto que nossos AGES III atrasaram a entrega, assim sobrecarregando
nossos AGES IV que cumpriram além de suas funções.
\par Dividimos-nos em squads, onde a minha possuía dois responsáveis
pelo backend e dois responsáveis pelo frontend. Fiquei no frontend com a
Carolina e demos inícios as tarefas.
\par Depois de alguns dias, começaram a surgir as tarefas e logo me
prontifiquei para realizar o desenvolvimento do componente de botão. De
certa forma foi fácil, apesar de ser uma primeira experiência minha com o
React na sua versão web, consegui já deixar pronto para os outros colegas
poderem utilizar em seus componentes.
\par Minha squad acabou ficando com uma User Story e demos início ao
desenvolvimento do componente de card de estudos. Este foi um pouco mais
complicado, visto que lidava com outros componentes e o processo de
estilização foi mais complexo. Aproveitei para criar o componente de Tag,
que serviria para nosso card e demos início ao desenvolvimento. Em poucos
dias estava pronto, faltava apenas a integração.
\par A integração foi uma fraqueza do grupo todo, visto que o frontend era
pouco conhecido pelos participantes, inclusive pelos AGES IV, então tivemos
uma aula de discussões e tentando entender como funcionaria todo o
processo de integração sem colocar a mão no código. A missão de casa e
pra última semana de sprint era entender como funcionava a integração.
\par Criei um componente de filtro de estudos, para utilizar com nosso
card. Mas ainda não sabia como integraria tudo, até que criei um
componente wrapper que “embrulhava” os dois, fazendo a chamada API no
próprio wrapper, que era utilizado pelo filtro, para depois ser disponibilizada
uma lista de estudos disponíveis, para então só depois colocar na página,
assim evitando que a página inteira fosse renderizada novamente a cada
interação.