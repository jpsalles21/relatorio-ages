\subsection{Sprint 3}

Durante a Sprint 3, minhas tarefas incluíam ajustes no funcionamento do filtro de estudos no botão de limpar, visto que ele não funcionava corretamente ao selecionar outro estudo. Também fiquei responsável para corrigir um erro na estilização da página de estudos que não tinha o comportamento esperado quando os cards de estudos apareciam na tela. Além disso, fiquei com uma tarefa extra, que só seria desenvolvida caso sobrasse tempo, que seria o início do desenvolvimento da tela de login.

Consegui implementar com sucesso o botão de limpar do filtro, que agora funciona conforme o esperado, e finalmente deixamos uma das partes mais importantes da aplicação completamente funcional e pronta para entrar na \textit{branch} principal. Para a correção de estilos foi bem simples, visto que era um problema de espaçamento e como eu já tinha experiência com HTML \cite{html} e CSS \cite{css}, foi a parte mais tranquila dessa sprint. Como ainda me sobrou uma semana para continaur desenvolvendo, decidi começar a tela de login, e combinei com meus colegas que apenas faria a parte estática, ou seja, qualquer tipo de integração com a api-auth seria completamente deixada de lado e inciada apenas na próxima sprint. Achei o modelo no figma e comecei os  trabalhos, o que não foi novamente muito difícil, visto que já estava habituado com as tecnologias. Aprendi como fazer uma funçõa de mostrar a senha, para que fosse possível caso o usuário desejasse visualizar sua senha antes de enviá-la, além de entender mais como funcionam os formulários. Esta sprint foi uma das melhores que tive no projeto, já que consegui adiantar todas minhas tarefas e ficar disponível caso alguém precisasse da minha ajuda, além de me sentir cada vez mais familiarizado com as tecnologias utilizadas. No fim, ficou pendente a parte de integração do login, que seria discutida no futuro com os AGES IV e AGES III. 