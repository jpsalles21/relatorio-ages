\subsection{Sprint 2}

Após apresentarmos para o cliente, ficou decidido que o filtro de estudos se transformaria em um \textit{dropdown}, além de uma reformulação completa na parte de integração, já que o \textit{frontend} adotara uma nova abordagem para realizar chamadas API. Inicialmente, imaginei que seria uma tarefa simples, pois já estava habituado com o filtro anterior. No entanto, logo percebi que a implementação seria mais desafiadora do que previsto. Decidimos criar um \textit{dropdown} com dois níveis, acompanhado por um botão para limpar as seleções e outro para filtrar e exibir os estudos.

Optamos por utilizar a biblioteca Material UI \cite{materialui}, e eu decidi empregar o \textit{React Select} para implementar o \textit{dropdown}. O início foi complicado, pois estava lidando com uma nova ferramenta. No entanto, a biblioteca se mostrou bastante intuitiva, o que facilitou a replicação do esperado. Consegui concluir a tarefa mais rápido do que o planejado. Contudo, enfrentamos desafios devido às enchentes que atingiram o estado do Rio Grande do Sul, resultando no cancelamento das aulas por duas semanas, seguidas por um retorno online. Por conta disso, fiquei um bom tempo afastado do projeto até a retomada da sprint e, em seguida, a entrega para o cliente. A entrega não saiu exatamente como esperávamos, mas conseguimos entregar o filtro quase completo, faltando apenas alguns ajustes básicos que não comprometiam o funcionamento da funcionalidade. Conforme previsto devido ao caos gerado pelas enchentes, alguns débitos técnicos foram deixados para serem resolvidos na próxima sprint. Apesar das dificuldades, a experiência acumulada com as bibliotecas utilizadas e a gestão de crises externas foi valiosa para a continuidade do projeto.