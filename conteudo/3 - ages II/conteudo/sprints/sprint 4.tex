\subsection{Sprint 4}

Na apresentação passada, o cliente nos informou que não poderíamos seguir utilizandos dados do paciente como nome, CPF e telefone. Com isso, nos reorganizamos e passamos a utilizar apelido, CEP e sexo. Em virtude do que foi mencionado, fiquei responsável pelo \textit{remap} dos dados do paciente no \textit{frontend}, tendo que mudar diversos arquivos. Comecei trocando nome por apelido, já que seria basicamente uma questão mais visual, já que ambos são simples \textit{strings}. Já para mudar CEP para CPF, foi um pouco mais trabalhoso, por contarmos com validadores e formatadores, o que no final foi excelente, fazendo com que eu trabalhasse com coisas novas e serviu de aprendizado para o futuro. Ao adicionar o sexo, imediatamente criei a opção de \textit{dropdown} na hora de cadastrar o pacientes, e estilizei novamente todo o componente, para que ficasse adequado. Partindo para a parte mais difícil, faltava integrar a tela de login com o sistema de autenticação. Foram horas de estudo, junto com o final de semestre, mas consegui fazer a autenticação funcionar atráves do \textit{Context API} do \textit{React}. A partir disso, criei uma Rota Privada, assim só sendo possível acessar qualquer página da aplicação se o \textit{token} fosse válido e estivesse ativo, caso contrário o usuário seria redirecionado para a tela de login novamente. Como não possuímos um meio de cadastro, toda essa parte foi feita por fora e de forma manual. No fim, esta sprint foi um pouco conturbada, pois ainda passávamos por consequências geradas pelas enchentes, já que o calendário acadêmico ficou mais curto e as provas e trabalhos acabaram coincidindo no final do semestre. Porém, mesmo com tudo isso, o trabalho feito foi excelente e toda a equipe conseguiu contribuir da maneira que dava. Fiquei muito feliz por aprender mais sobre \textit{keycloack} e maneiras de realizar uma autenticação, além de trabalhar pela primeira vez com docker \cite{docker}.