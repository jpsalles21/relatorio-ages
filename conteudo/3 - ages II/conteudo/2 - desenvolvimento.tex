\section[Desenvolvimento do Projeto]{Desenvolvimento do Projeto}

\subsection{Repositório do Código Fonte do Projeto}
  Os repositório foram divididos entre web-medico, web-site, web-administrador,
api-medico, api-administrador, auth, infra e wiki.
\par Repositório para o \href{https://tools.ages.pucrs.br/let-me-trial/web-site}{web-site}
\par Repositório para o \href{https://tools.ages.pucrs.br/let-me-trial/webmedico}{web-medico}
\par Repositório para  o \href{https://tools.ages.pucrs.br/let-me-trial/webadministrador}{web-administrador}
\par Repositório para o \href{https://tools.ages.pucrs.br/let-me-trial/api-medico}{api-medico}{}
\par Repositório para \href{https://tools.ages.pucrs.br/let-me-trial/api-auth}{auth}
\par Repositório para \href{https://tools.ages.pucrs.br/let-me-trial/infra}{infra}
\par Repositório para wiki: \href{https://tools.ages.pucrs.br/let-me-trial/wiki}{wiki}

\subsection{Banco de Dados Utilizado}
No Banco de Dados, optamos por utilizar o PostgresSQL pela sua robustez.
Entidades do projeto
Paciente: Armazena as informações pertinentes ao projeto do paciente
Medico: Armazena as informações pertinentes ao projeto do médico.
Area: Armazena as subareas da qual pertencem os estudos.
Resposta: Armazena a resposta de um paciente sobre um determinado critério.
Estudo: Armazena informações de estudos e os critérios pertencentes através da entidade CritérioEstudo.
Criterio: Armazena a pergunta sobre um critério.
CriterioEstudo: Armazena os dados sobre um critério referente a um determinado estudo, como resposta esperada por ele, se é um critério opcional e relaciona o critério ao estudo.`

\subsection{Arquitetura Utilizada}
 No backend, estamos utilizando a Arquitetura Hexagonal, que é um modelo
de design de software que divide o sistema em três partes principais:
Domínio: o coração do sistema, onde estão as regras de negócio e entidades
importantes. Aplicação: coordena as ações do sistema e lida com interações entre o
domínio e o mundo externo. Adaptadores: conecta o sistema ao mundo externo,
permitindo interações sem que o domínio precise saber dos detalhes técnicos.
No frontend, utilizamos o núcleo/core e seus submódulos, como core/domain para
definir as entidades principais da aplicação, como usuários e médicos. O
core/useCases é essencial para dividir a lógica de negócios em casos de uso
13
comuns e específicos para diferentes tipos de usuários, enquanto o
adapters/presenter nos ajuda a adaptar os dados da lógica de negócios para a
interface do usuário. Os frameworks/ui são vitais para organizar os componentes e
páginas da interface do usuário, enquanto os frameworks/services facilitam a
comunicação com APIs externas. Por fim, os utils são úteis para funções utilitárias,
como autenticação e verificação do tipo de usuário

\subsection{Protótipos das Telas Desenvolvidas}
\begin{figure}[H]
    \centering
    \small
    \caption{Protótipo no Figma}
    \includegraphics[width=1\linewidth]{conteudo/3 - ages II/conteudo/figures/figma2.png}
    Fonte: figma do projeto
    \label{fig:projeto-time}
\end{figure}

\subsection{Tecnologias Utilizadas}
  No backend, estamos utilizando Java 21 \cite{java} com o framework Spring Boot \cite{springboot}, além
do JUnit para testes, pois, a maioria do grupo era familiarizado com a linguagem, o
que facilitaria e aumentaria a velocidade e a qualidade do projeto. No frontend
optamos por utilizar a biblioteca React, com TypeScript e o framework Next. Para o
frontend, utilizo o Postman \cite{postman} para testar as APIs antes de alguma integração.
  
  