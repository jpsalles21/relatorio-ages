\section[Conclusão]{Conclusão}

O projeto Se Doce Fosse representou um marco significativo na minha trajetória como desenvolvedor de software, proporcionando um ambiente desafiador e enriquecedor para aplicar e expandir meus conhecimentos técnicos, já que dessa vez atuei como Arquiteto de Software e Líder técnico do Frontend. Entender melhor sobre infraesturura e realizar um deploy na AWS foram experiências valiosas que complementaram minha formação e me prepararam para desafios futuros na área de desenvolvimento de software. Durante o projeto, enfrentei diversos desafios técnicos e de comunicação que exigiram soluções criativas e colaborativas. A necessidade de alinhar as expectativas entre as equipes de frontend e backend, bem como a gestão de tarefas entre os diferentes níveis de experiência dos membros do time, foram aspectos cruciais que contribuíram para meu crescimento profissional.
\par Sobre o projeto em si, acredito que consegui contribuir de maneira significativa para o sucesso das entregas, especialmente ao apoiar os AGES I e AGES II no frontend e na resolução de bloqueios técnicos, minha maior meta era fazer com que o aprendizado deles não fosse deixado de lado, pois como já visto na primeira reunião, eles nunca tinham tido contato com esse tipo de tecnologia antes e creio que a AGES não é só sobre entregar um projeto, mas sim sobre o aprendizado e crescimento de todos os envolvidos. Um dos maiores problemas era ter um número alto de AGES IV, já que isso acabava por nenhum deles ter o controle total do projeto, fazendo com que um dependesse do outro para resolver problemas, o que não é o ideal, já que precisamos tomar deciões mais assertivas o quanto antes. Esse problema seria solucionado se eles mesmo tivessem sido designados cada um para um tipo de tarefa, como por exemplo, um para infraestrutura, outro para backend e outro para frontend, assim cada um teria total controle sobre sua área e poderia tomar decisões mais rápidas e eficazes.
\par Em resumo, o projeto Se Doce Fosse foi uma experiência transformadora que ampliou meus horizontes técnicos e interpessoais. As lições aprendidas sobre liderança técnica fizeram com que eu me tornasse um profissional mais completo e preparado para enfrentar os desafios do desenvolvimento de software em equipe. Ademais, a oportunidade de ensinar alguém com pair programming foi extremamente gratificante, pois pude ver o crescimento dos colegas e a evolução do projeto como um todo, já que no final, os AGES I estavam realizando tarefas de maneira autonôma, o que demonstra o sucesso do processo de aprendizado e colaboração implementado ao longo do projeto, que era minha meta inicial.