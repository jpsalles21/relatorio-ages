\section[Desenvolvimento do Projeto]{Desenvolvimento do Projeto}

\subsection{Repositório do Código Fonte do Projeto}
Os repositórios foram definidos em três principais:
\par Repositório para o frontend: \href{https://tools.ages.pucrs.br/se-doce-fosse/frontend}{https://tools.ages.pucrs.br/se-doce-fosse/frontend}, onde colocamos toda a parte web do projeto. Esse repositório contém a aplicação React com a organização de componentes, estilos e scripts de build, um README com instruções de execução local e exemplos de dados para testes. Utilizamos branches para separar `main` (release) de `develop` (integração) e GitHub Actions para pipeline de build e testes.
\par Repositório para o backend: \href{https://tools.ages.pucrs.br/se-doce-fosse/backend}{https://tools.ages.pucrs.br/se-doce-fosse/backend}, onde colocamos a lógica de negócio com as APIs e banco de dados. Aqui ficam o projeto Spring Boot, as configurações que usam variáveis de ambiente (com um arquivo de exemplo), as migrações/instruções dos bancos e os workflows de CI/CD. O fluxo de trabalho é baseado em pull requests para revisão e issues para acompanhamento das tarefas.
\par Repositório para a wiki: \href{https://tools.ages.pucrs.br/se-doce-fosse/wiki}{https://tools.ages.pucrs.br/se-doce-fosse/wiki}, onde documentamos tudo que envolva o projeto de uma forma mais didática. A wiki reúnxe rotas da API, modelos de dados, instruções de deploy, convenções de codificação e guias rápidos de uso, servindo como referência para integrantes com diferentes níveis de experiência e quem for dar manutenção no futuro.



\subsection{Banco de Dados Utilizado}
No desenvolvimento do sistema, optou-se por uma abordagem dupla de bancos de dados, utilizando o PostgreSQL e o MongoDB, cada um desempenhando papéis complementares. O PostgreSQL \cite{postgresql}, banco de dados relacional amplamente reconhecido por sua robustez e confiabilidade, foi empregado para armazenar informações que exigem alta integridade e consistência, como dados de usuários, controle de estoque e histórico de pedidos. Já o MongoDB \cite{mongodb}, banco de dados não relacional orientado a documentos, foi escolhido para gerenciar elementos mais dinâmicos, como o carrinho de compras e listas temporárias, permitindo maior flexibilidade na estruturação e manipulação dessas informações. Essa combinação possibilita um equilíbrio entre desempenho, escalabilidade e integridade, explorando as vantagens específicas de cada tecnologia.
  
\begin{figure}[H]
    \centering
    \small
    \caption{Banco de Dados}
    \includegraphics[width=1\linewidth]{conteudo/4 - ages III/conteudo/figures/BD_SeDosseFosse.png}
    Fonte: Wiki do Projeto
    \label{fig:projeto-time-se-doce-fosse}
\end{figure}


\subsection{Arquitetura Utilizada}
No projeto, adotou-se uma arquitetura em camadas (\textit{Layered Architecture}) para o desenvolvimento do backend, implementado em Java \cite{java} com o framework Spring Boot \cite{springboot}. Essa organização separa claramente as responsabilidades entre as camadas de apresentação, serviços, repositórios e entidades, promovendo maior manutenibilidade, testabilidade e clareza no fluxo de dados. No frontend, foi utilizada a biblioteca React \cite{react}, estruturada de forma modular, permitindo o reaproveitamento de componentes e a escalabilidade da aplicação. Para o estilo, optou-se pelo uso de SCSS Modules, que garantem isolamento de escopo e facilitam a manutenção visual da interface, contribuindo para uma organização consistente entre lógica, apresentação e estilos em toda a aplicação. Na parte de Infraestrutura, utilizamos a \ac{aws} \cite{aws} para hospedar a aplicação, criando uma EC2 para o frontend, uma para o backend junto dos bancos de dados. O frontend utiliza NGINX para servir arquivos estáticos, atuar como reverse proxy para o backend, realizar terminação SSL/TLS, fazer cache e compressão, e gerenciar balanceamento de carga quando necessário, garantindo desempenho e segurança. Para hospedar imagens, utilizamos o Amazon S3, que oferece armazenamento escalável e seguro para os arquivos estáticos do projeto, como imagens de produtos. Também utilizamos o GitHub Actions para realizar o CI/CD do projeto, automatizando os processos de build e testes.

    \begin{figure}[H]
    \centering
    \small
    \caption{Diagrama de Arquitetura}
    \includegraphics[width=1\linewidth]{conteudo/4 - ages III/conteudo/figures/diagrama-sedocefosse.png}
    Fonte: Wiki do Projeto
    \label{fig:projeto-time-se-doce-fosse}
\end{figure}

  
\subsection{Protótipos das Telas Desenvolvidas}
  Na Sprint 0, protótipos de telas foram desenvolvidos e entregues aos clientes. Usamos elas como base para construir o frontend e validar com o clientes, além de auxiliar no backend, para entender quais endpoints seriam necessários. O protótipo foi desenvolvido no Figma, uma ferramenta colaborativa de design de interfaces que permite a criação de protótipos interativos e compartilháveis. A seguir, imagens do protótipo desenvolvido:
  \begin{figure}[H]
    \centering
    \small
    \caption{Protótipo no Figma - Home}
    \includegraphics[width=1\linewidth]{conteudo/4 - ages III/conteudo/figures/FigmaAgesIII.jpg}
    Fonte: Figma do projeto
    \label{fig:projeto-time-se-doce-fosse}
\end{figure}

\begin{figure}[H]
   \centering
   \small
   \caption{Protótipo no Figma - Página de Produtos}
   \includegraphics[width=1\linewidth]{conteudo/4 - ages III/conteudo/figures/FigmaProdutos.jpg}
   Fonte: Figma do Projeto
   \label{fig:projeto-time-se-doce-fosse}
\end{figure}

 \begin{figure}[H]
    \centering
    \small
    \caption{Protótipo no Figma - Tabela de pedidos (Admin)}
    \includegraphics[width=1\linewidth]{conteudo/4 - ages III/conteudo/figures/FigmaPedidosAdmin.jpg}
    Fonte: Figma do Projeto
    \label{fig:projeto-time-se-doce-fosse}
\end{figure}

  \begin{figure}[H]
    \centering
    \small
    \caption{Protótipo no Figma - Página Admin}
    \includegraphics[width=1\linewidth]{conteudo/4 - ages III/conteudo/figures/FigmaAdmin.jpg}
    Fonte: Figma do projeto
    \label{fig:projeto-time-se-doce-fosse}
\end{figure}





\subsection{Tecnologias Utilizadas}
  Escolhemos uma stack que equilibra aprendizado e robustez: Java 21 \cite{java} com Spring Boot \cite{springboot} no backend, PostgreSQL \cite{postgresql} para dados relacionais críticos e MongoDB \cite{mongodb} para dados mais flexíveis, e React com TypeScript \cite{react,typescript} no frontend.

  Java e Spring Boot foram adotados pela familiaridade dos alunos e pela produtividade na criação de APIs testáveis. PostgreSQL garante integridade e consultas complexas, enquanto MongoDB facilita dados com esquema variável (ex.: carrinho). React + TypeScript oferece componentes reutilizáveis e verificação de tipos em desenvolvimento. Complementamos com SCSS Modules para estilos, NGINX para servir o frontend e GitHub Actions para CI/CD. A infraestrutura utiliza AWS \cite{aws} (S3 e EC2) quando necessário. Essa combinação também expõe os alunos a paradigmas diferentes (relacional e documental) sem aumentar muito a complexidade operacional.

  Essas escolhas priorizam aprendizado prático e entrega eficiente, evitando complexidade desnecessária ao mesmo tempo em que expõem os alunos a tecnologias e padrões usados em produção. Além disso, priorizamos práticas como testes automatizados e integração contínua para reforçar formação profissional e preparar os alunos para contextos reais de desenvolvimento.


  

