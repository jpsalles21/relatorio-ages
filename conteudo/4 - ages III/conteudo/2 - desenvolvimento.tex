\section[Desenvolvimento do Projeto]{Desenvolvimento do Projeto}

\subsection{Repositório do Código Fonte do Projeto}
Os repositórios foram definidos em 3 conteúdos sendo eles:
\par Repositório para o \href{https://tools.ages.pucrs.br/se-doce-fosse/frontend}{frontend}, onde colocamos toda a parte web do projeto.
\par Repositório para o \href{https://tools.ages.pucrs.br/se-doce-fosse/backend}{backend}, onde colocamos a lógica de negócio com as APIs e banco de dados.
\par Repositório para a \href{https://tools.ages.pucrs.br/se-doce-fosse/wiki}{wiki}, onde documentamos tudo que envolva o projeto de uma forma mais didática.



\subsection{Banco de Dados Utilizado}
No desenvolvimento do sistema, optou-se por uma abordagem dupla de bancos de dados, utilizando o PostgreSQL e o MongoDB, cada um desempenhando papéis complementares. O PostgreSQL \cite{postgresql}, banco de dados relacional amplamente reconhecido por sua robustez e confiabilidade, foi empregado para armazenar informações que exigem alta integridade e consistência, como dados de usuários, controle de estoque e histórico de pedidos. Já o MongoDB \cite{mongodb}, banco de dados não relacional orientado a documentos, foi escolhido para gerenciar elementos mais dinâmicos, como o carrinho de compras e listas temporárias, permitindo maior flexibilidade na estruturação e manipulação dessas informações. Essa combinação possibilita um equilíbrio entre desempenho, escalabilidade e integridade, explorando as vantagens específicas de cada tecnologia.
\subsection{Arquitetura Utilizada}
No projeto, adotou-se uma arquitetura em camadas (\textit{Layered Architecture}) para o desenvolvimento do backend, implementado em Java \cite{java} com o framework Spring Boot \cite{springboot}. Essa organização separa claramente as responsabilidades entre as camadas de apresentação, serviços, repositórios e entidades, promovendo maior manutenibilidade, testabilidade e clareza no fluxo de dados. No frontend, foi utilizada a biblioteca React \cite{react}, estruturada de forma modular, permitindo o reaproveitamento de componentes e a escalabilidade da aplicação. Para o estilo, optou-se pelo uso de SCSS Modules, que garantem isolamento de escopo e facilitam a manutenção visual da interface, contribuindo para uma organização consistente entre lógica, apresentação e estilos em toda a aplicação.
\subsection{Protótipos das Telas Desenvolvidas}
  Na Sprint 0, protótipos de telas foram desenvolvidos e entregues aos clientes. Usamos elas como base para construir o frontend.
  \begin{figure}[H]
    \centering
    \small
    \caption{Protótipo no Figma}
    \includegraphics[width=1\linewidth]{conteudo/4 - ages III/conteudo/figures/FigmaAgesIII.jpg}
    Fonte: figma do projeto
    \label{fig:projeto-time-se-doce-fosse}
\end{figure}


\subsection{Tecnologias Utilizadas}
  Estamos utilizando uma stack versátil que conta com Java 21 \cite{java} com Spring Boot \cite{springboot} no backend, com dois bancos: PostgreSQL \cite{postgresql} e MongoDB \cite{mongodb}. No frontend, estamos utilizando TypeScript \cite{typescript} com React \cite{react}. 
  A escolha de Java 21 e SpringBoot, se define pela facilidade de aprendizado dos alunos, já que a faculdade introduz uma base forte e muito boa em Java, além de contar com a robustez que o framework Spring Boot traz, sendo uma excelente escolha para desenvolver o projeto. 
  No frontend, optamos por React apenas com TypeScript. React base, sem qualquer outro framework, para extrair apenas o necessário e não contar com overengineering, já que o projeto não demanda de muitas caractersticas, além de ajudar a curva de aprendizado dos AGES I. O TypeScript foi escolhido para dar uma camada de tipos, para garantir mais segurança e boas práticas de programação.

