\subsection{Sprint 1}

Na Sprint 1, foquei em organizar o trabalho do frontend e em dar suporte aos AGES I e II, garantindo que todos pudessem avançar sem bloqueios. Logo no início, separei as tarefas de acordo com o nível de cada integrante, definindo o que seria mais adequado para os AGES I e para os AGES II. Essa divisão ajudou bastante a dar clareza sobre as responsabilidades e permitiu que o time tivesse uma visão mais concreta do que precisava ser entregue.

Além da organização das tasks, trabalhei diretamente na implementação de alguns componentes essenciais para o projeto, como o \textit{Input} e o \textit{Textarea}. Esses elementos serviram como base para o restante da interface e já deixaram o repositório com exemplos práticos para os outros integrantes, e também já deixava pronto esses componentes que seriam reutilizados em tarefas futuras. Também revisei, comentei e aprovei diversos Pull Requests, ajudando a manter a qualidade e a consistência do código que estava sendo incorporado ao projeto.

Outro ponto marcante dessa sprint foi o tempo dedicado a desimpedir colegas que estavam com dificuldades para entender o funcionamento do frontend. Muitos dos integrantes, especialmente os do AGES I, estavam tendo contato com essas tecnologias pela primeira vez e se sentiram um pouco perdidos. Para ajudar, organizei “aulas” no Discord, ferramenta que utilizamos como base para comunicação, onde expliquei o passo a passo de como os componentes funcionam, apresentei a estrutura do projeto e mostrei boas práticas para a escrita de código. Em vários momentos também fiz \textit{pair programming}, acompanhando o desenvolvimento em tempo real e orientando sobre a melhor forma de resolver problemas específicos, além de ajudar a entender funções específicas do TypeScript.

Percebi que esse apoio constante fez diferença no engajamento do grupo. À medida que as dúvidas eram resolvidas, os colegas conseguiam evoluir mais rápido e participar ativamente das entregas. Ainda assim, notei que apenas explicações ao vivo não eram suficientes para todos, por isso, comecei a pensar na ideia de criar materiais em vídeo que complementassem a documentação já disponível, tornando o aprendizado mais acessível e facilitando o trabalho de quem prefere revisar os conteúdos de outra maneira.

Enquanto essas atividades aconteciam, também iniciei estudos sobre a infraestrutura que pretendemos usar na AWS. Esse planejamento começou de forma paralela, para que as decisões sobre deploy e hospedagem não atrasem o projeto nas próximas sprints.

De forma geral, a Sprint 1 foi muito voltada a facilitar o caminho dos colegas e a consolidar a base do frontend. Consegui equilibrar a produção de código com o suporte técnico, garantindo que todos tivessem condições de avançar. O trabalho de orientação, revisão e organização foi essencial para criar um ambiente colaborativo e produtivo, preparando o time para as próximas etapas do desenvolvimento.