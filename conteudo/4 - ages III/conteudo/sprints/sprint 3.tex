\subsection{Sprint 3}
 
Na Sprint 3, novamente concentrei meus esforços em desimpedir os AGES I e II, atuando tanto na parte de API quanto no código do frontend. Não foi necessário gravar o vídeo previsto para as integrações, pois os colegas compreenderam rapidamente os conceitos após as explicações, exemplos práticos e sessões de acompanhamento que realizei. Assumi tarefas de integração entre frontend e backend: padronizei as chamadas, ajustei formatos de dados quando necessário e validei fluxos básicos com ferramentas de teste (ex.: requisições manuais e mocks), o que permitiu que várias funcionalidades passassem a se comunicar corretamente com a API. Embora o deploy final não tenha sido concluído nesta sprint, avançamos bastante no planejamento da entrega e em testes locais colaborativos com os demais dos AGES III.

Durante o processo enfrentei problemas de integração relevantes: o backend ainda apresentava instabilidades e, em alguns momentos, alterações em campos da tabela de produtos não eram persistidas, impedindo a finalização de funcionalidades no frontend. Para identificar a causa, conversei com a equipe de backend, inspecionei logs, testei endpoints isoladamente e adotei verificações intermediárias no frontend que evidenciaram quais valores não chegavam corretamente ao banco. Também houve dependência de decisões sobre reorganização dos bancos de dados, o que exigiu ajustes na arquitetura a serem feitos antes do deploy definitivo.

Além disso, realizei pair programming com AGES I e revisei pull requests para manter a qualidade do código. Essas ações reduziram o impacto dos problemas e mantiveram o ritmo das entregas, além de gerar material e exemplos que facilitaram o aprendizado dos colegas.

As lições aprendidas nesta sprint reforçaram que frontend e backend devem caminhar mais alinhados desde o início, com contratos de API claros e testes de integração contínuos. A prática de escrever e executar testes, mesmo simples e manuais nas fases iniciais, ajudaria a detectar inconsistências e evitar retrabalhos que serão feitos. Em suma, a Sprint 3 foi produtiva para desbloquear o time, consolidar integrações e avançar no planejamento do deploy, as pendências remanescentes servem agora como pontos de ação claros para as próximas sprints.