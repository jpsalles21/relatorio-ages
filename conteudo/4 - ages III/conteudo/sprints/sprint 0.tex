\subsection{Sprint 0}

Durante a Sprint 0, meu trabalho esteve totalmente voltado para preparar o ambiente necessário ao desenvolvimento do projeto, garantindo que toda a equipe tivesse as condições ideais para iniciar as próximas etapas. A primeira ação foi a inicialização do repositório principal, criando uma base organizada para armazenar o código-fonte do frontend e demais materiais relacionados ao sistema. Além disso, adicionei todos os membros da equipe aos repositórios, concedendo as permissões adequadas para que cada integrante pudesse colaborar de forma integrada desde o início, além de estabelecer regras para protegers as branchs.

Também realizei a dockerização do ambiente do frontend, criando uma estrutura que permitisse aos integrantes responsáveis pelo backend testar a interface de forma simples e rápida, sem precisar configurar dependências locais. Configurei os arquivos e parâmetros do Docker com esse objetivo, garantindo que o front pudesse ser executado em qualquer máquina de maneira padronizada, apenas subindo o container. Essa etapa foi essencial para reduzir barreiras entre as áreas, facilitar a integração e evitar problemas relacionados a diferenças de versões ou de configuração.

Paralelamente às tarefas mais técnicas, estudei as tecnologias ligadas à infraestrutura que seriam utilizadas ao longo do projeto. Também dediquei tempo para estruturar e documentar os processos que a equipe deveria seguir. Criei uma seção específica na Wiki chamada “Processos”, onde registrei todas as definições acordadas, incluindo fluxos de trabalho, regras para criação de branches, padrões de commits e boas práticas para organização do código. O objetivo foi garantir que ninguém tivesse dúvidas sobre como proceder, permitindo que todos soubessem exatamente quais passos adotar no dia a dia.

Ao longo dessa sprint, mantive um olhar atento para assegurar que cada detalhe estivesse finalizado antes do encerramento do ciclo. Verifiquei o funcionamento do espelhamento entre as plataformas, testei os containers Docker para confirmar que estavam executando corretamente e revisei as permissões dos membros nos repositórios. Além disso, organizei os arquivos iniciais do projeto, deixando tudo estruturado para que a equipe pudesse começar a desenvolver as funcionalidades planejadas sem enfrentar bloqueios relacionados à configuração do ambiente.

Concluindo, a Sprint 0 foi um período dedicado a preparar a base técnica e organizacional do projeto. As atividades que realizei criaram uma base sólida para o trabalho das próximas sprints. Essa preparação garantiu que o time tivesse clareza sobre as práticas adotadas e que todos dispusessem de um ambiente uniforme, seguro e pronto para receber as implementações do sistema.
